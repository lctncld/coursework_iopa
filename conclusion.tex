%!TEX root = main.tex

\newpage
%\section*{ЗАКЛЮЧЕНИЕ}
%\addcontentsline{toc}{section}{ЗАКЛЮЧЕНИЕ}
\begin{center}
\textbf{ЗАКЛЮЧЕНИЕ}
\end{center}
\addcontentsline{toc}{section}{ЗАКЛЮЧЕНИЕ}

В результате выполнения курсовой работы был синтезирован автомат, реализующий две операции: $A \leftarrow A - 1|_{8421+3}$ (арифметическую) и $A \leftarrow A \& B$ (логическую), устанавливающий флаги $S, Z, P, C, C'$ в зависимости от результата операции.

При синтезе автомат ОА был представлен в виде двух автоматов: ОА$_1$ и ОА$_2$. Первый автомат осуществляет выполнение операции, второй --- устанавливает флаги признаков.

Автомат ОА$_1$ был декомпозирован на два автомата: ОА$^{(0)}_{1}$ и ОА$^{(1)}_{1}$. 

ОА$^{(0)}_{1}$ выполняет операцию $A \leftarrow A - 1|_{8421+3}$ и вырабатывает признаки результата на основе последующего состояния $A(t+1)$. Этот автомат был представлен как единый 4-разрядный ОЭ.

ОА$^{(1)}_{1}$ выполняет операцию $A \leftarrow A \& B$ и вырабатывает признаки результата на основе текущего состояния $A(t)$. Этот автомат был представлен как композиция одноразрядных ОЭ.

Также было проведено моделирование полученного автомата с помощью САПР <<Альтера>> Max+plus II.
